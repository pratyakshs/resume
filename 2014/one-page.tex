
\documentclass{resume2} % Use the custom resume.cls style
\usepackage{graphicx}
\usepackage{enumitem}
\usepackage[left=0.5in,top=0.4in,right=0.5in,bottom=0.4in]{geometry} % Document margins
\linespread{0.85}
\usepackage{setspace}
\usepackage{enumitem}
\usepackage{array}
\usepackage{hyperref}

\begin{document}

\vspace{1.3cm}


\begin{rSection}{Scholastic Achievements}
\begin{itemize}[leftmargin=*]
\item[$\star$] \textbf{All India Rank 33} in \textbf{IIT Joint Entrance Examination} (2012) out of over 500,000 candidates
\item[$\star$] \textbf{All India Rank 29} in \textbf{AIEEE} (2012) out of over 1,200,000 candidates
\item[$\star$] Qualified for \textbf{Indian National Physics (INPhO), Chemistry (INChO) and Astronomy (INAO) Olympiads} (2012) (\textbf{top 300} students in India in each subject) and awarded certificates of merit for being in \textbf{top 1\%} students
\item[$\star$] Qualified for the \textbf{Indian National Mathematics Olympiad} after clearing RMO in Delhi (2011)
\item[$\star$] Awarded the \textbf{Kishore Vaigyanik Protsahan Yojana} (KVPY) scholarship by Department of Science and Technology of the Government of India, aimed at encouraging students to take up research careers
\item[$\star$] \textbf{Honourable Mention} at \textbf{ACM-ICPC} Amritapuri Regionals, Onsite Rank-24, Online Rank-17
\item[$\star$] Admitted to the \textbf{Education Program for Gifted Youth} (EPGY) Summer
Institutes High School Program at \textbf{Stanford University} (2010)
\item[$\star$] Awarded \textbf{Gold medal} by DPS Society for maintaining excellent academic record for seven consecutive years
\end{itemize}
\end{rSection}

%-------------

\begin{rSection}{Projects and Internships}

\begin{rSubsection}{\href{http://www.ics.uci.edu/~pratyas/report2014.pdf}{Inference in Probabilistic Graphical Models}}{May 2014 - Jul 2014}{Guide: Prof. Rina Dechter (University of California, Irvine)
}{}
\item[$\star$] Performed empirical analysis of \textbf{anytime weighted best first strategies on AND/OR search spaces} for approximate inference 

\item[$\star$] Participated in \textbf{UAI 2014 - Probabilistic Inference Competition}
\item[$\star$] Currently studying \textbf{weighted schemes with dynamic weights}, which may lead to a smaller search space
\item[$\star$] \emph{Url:} \url{http://www.ics.uci.edu/~pratyas/report2014.pdf}
\end{rSubsection}


\begin{rSubsection}{Sentiment analysis of song lyrics}{Aug 2014 - Present}{Guide: Prof. Pushpak Bhattacharyya, IIT Bombay}{}
\item[$\star$] Developing a system to analyse song lyrics and output the mood/emotion/sentiment associated with the song
\item[$\star$] Classifying a song solely based on its lyrics is challenging. All current algorithms have poor accuracy
\end{rSubsection}


\begin{rSubsection}{\href{https://github.com/pratyakshs/Che.ss}{Chess}}{Jan 2013 - Apr 2013}{Guide: Prof. Amitabha Sanyal, IIT Bombay}{}
\item[$\star$] Developed a chess engine in the \textbf{functional programming} paradigm using PLT Scheme
\item[$\star$] The \textbf{minimax algorithm} with \textbf{alpha-beta pruning} formed the basis of the AI and also investigated other chess algorithms like \textbf{Negascout} and \textbf{MTD-f} \item[$\star$] The heuristics involved use of Piece-Square tables, analyzing pawn structure and various other chess tactics.
\item[$\star$] Used the \textbf{XBoard} GUI, and made a parser for the \textbf{Chess Engine Communication Protocol}
\item[$\star$] \emph{Url:} \url{https://github.com/pratyakshs/Che.ss}
\end{rSubsection}




%\begin{rSubsection}{Android PC controller}{May 2013 - Jun 2013}{Institute Technical Summer Project}{}
%\item[$\star$] Developed, in a team of four, an \textbf{Android application} to control a PC, using a mobile device
%\item[$\star$] Studied Android \textbf{bluetooth socket programming} and used Java's Robot class
%\end{rSubsection}

%\begin{rSubsection}{Pac-man}{Aug 2012 - Nov 2012}{Guide: Prof. Abhiram Ranade (CSE Dept., IIT Bombay)
%}{}
%\item[$\star$] Recreated the classic game of pac-man using \textbf{Object Oriented Programming} in C++ using the EzWindows graphics library.
%\item[$\star$] Programmed the deterministic ghost behaviour as in the standard game of pac-man
%\end{rSubsection}


\begin{rSubsection}{Analytics developer}{Jul 2013 - Sep 2013}{\href{http://coursewave.org/}{Coursewave.org}, an online courseware startup}{}
\item[$\star$] Developed live data visualizations for students and teachers using various web APIs
\item[$\star$] Integrated the analytics with online test taking interface
\end{rSubsection}

\end{rSection}

%--------------

\begin{rSection}{Technical Skills}

\begin{tabular}{ @{} >{\bfseries}l @{\hspace{6ex}} l }
Computer Languages & C/C++, Python, PLT Scheme, Prolog, Java, Bash, VHDL, MIPS\\
Web \& APIs & HTML, CSS, Javascript, PHP, MySQL, AJAX, JSON \\
Tools \& Libraries &  Sage, \LaTeXe , Scilab, PSPICE, Mathematica, Numpy, Boost C++
\end{tabular}
\\
\end{rSection}



\end{document}
